分数阶导数的记忆性导致数值方法的收敛性难以分析。针对这一问题尚未找到有效的解决办法。给定区间$[0,T]$, 其中$0<T<\infty$, 设$x$是这一区间上方程的准确解(弱解),并记由L1格式\eqref{L1格式}在时间序列$\left(t_k\right)_{k=0}^n$下产生的数值解为$\left(x_k\right)_{k=0}^n$, 其中$0=t_0<t_1<t_2<\dots<t_n=T$, 而考察收敛性实际上是考察$\max_{1\leqslant k\leqslant n} \left(t_k-t_{k-1}\right)$趋于0时是否有$x(T)-x_n$趋于0, 注意这里必定伴随着$n\to \infty$. 分数阶导数的记忆性导致累积误差难以控制。我会继续查阅相关资料,并积极与老师同学讨论、交流思路和成果;也会从其他一些问题中寻求灵感,比如导数阶数大于1的情形。