设$d\in \positiveinteger$. 本课题主要研究如下的分数阶比例延迟方程,
\begin{equation}\label{方程}
    \begin{cases}%dcases
        \Caputo x(t) = f(t, x(t), x(qt)),&t\geqslant 0,\\
        x(0) = x_0,
    \end{cases}
\end{equation}
其中$x$是$\realset^d$值未知函数,$x_0\in \realset^d$给定,$0<\alpha<1$, 函数$f:[0,\infty)\times \realset^d\times \realset^d\to\realset^d$连续,$\Caputo x(t)=\Gamma(1-\alpha)^{-1}\int_{0}^{t}(t-s)^{-\alpha}x'(s)\differential s$. 参考Webb J. R. L.的文献\inlinecite{mild}, 我们给出对于\mainEquation 的弱解的定义。
\begin{definition}
    如果有函数$x$, 使得$t$属于$[0,\infty)$的某个包含0的子区间$I$上时满足
\begin{equation}\label{mild solution}
    x(t)=x_0+\Gamma(\alpha)^{-1}\int_{0}^{t} (t-s)^{\alpha-1} f(s,x(s),x(qs)) \differential s,
\end{equation}
那么称$x$是\mainEquation 在$I$上的一个弱解。
\end{definition}
目前研究了弱解的Peano存在性和Picard存在唯一性定理,并对任意步长下的L1数值格式的长时间行为做了讨论。

在准确解部分得到的结果见如下的\cref{Peano,Picard}\hspace{-.25em}.
\begin{theorem}[Peano存在性定理]\label[theorem]{Peano}
    \mainEquation 总是在某个小区间$[0,h]$上存在弱解。
\end{theorem}
\begin{theorem}[Picard存在唯一性定理]\label[theorem]{Picard}
    如果$f(t,\cdot,\cdot)$对$t\in [0,\infty)$一致地局部Lipschitz, 即对任何$r>0$, 存在不依赖于$t$的$L=L(r)\geqslant 0$, 使得
    \begin{equation}\label{Lipschitz}
        \| f(t,x,y) - f(t,u,v) \| \leqslant L\cdot (\|x-u\| + \|y-v\|)
    \end{equation}
    对任何$t\in [0,\infty)$以及$x,y,u,v\in B_r(0)$成立,那么\mainEquation 在某个小区间$[0,h]$上存在弱解,并且弱解在存在区间$I\ni 0$上唯一。进一步地,如果$L$可以不依赖于$r$, 那么在$[0,\infty)$上全局存在唯一的弱解。
\end{theorem}

在叙述数值解部分的结果之前,先给出数值格式。取严格递增趋于正无穷的序列$\left(t_n\right)_{n=0}^\infty$作为时间节点,其中$t_0=0$. 利用在每个小区间上线性插值的办法来近似导数和延迟,就得到针对\mainEquation 的L1数值格式
\begin{equation}\label{L1格式}
    x_n=\left(t_n-t_{n-1}\right)^\alpha \Bigl(\Gamma(2-\alpha) f\left(t_n,x_n,\overline{x}^n\right) + \sum_{k=0}^{n-1} \left(a_{n,k+1}-a_{n,k}\right) x_k\Bigr),
\end{equation}
其中$a_{n,k}:=\frac{\left(t_n-t_{k-1}\right)^{1-\alpha}-\left(t_n-t_{k}\right)^{1-\alpha}}{t_k-t_{k-1}} (1\leqslant k\leqslant n),a_{n,0}:=0,\overline x_n$是对$x(qt_n)$的近似,即设$t_{m_n-1}\leqslant qt_n<t_{m_n}$, 有
\begin{equation*}
    \overline x_n:=\frac{t_{m_n}-qt_n}{t_{m_n}-t_{m_n-1}}x_{m_n-1}+\frac{qt_n-t_{m_n-1}}{t_{m_n}-t_{m_n-1}}x_{m_n}.
\end{equation*}

在数值解部分得到的结果为如下的\cref{dissipativity,Picard}, 其中所描述的数值解的长时间行为与文献\inlinecite{WangDongLing}中得到的准确解在相同条件下的长时间行为相吻合,同时也是对文献\inlinecite{LiDongFang}中结果的推广。接下来的\cref{dissipativity}说明,在一定条件下,数值解有界。
\begin{theorem}\label{dissipativity}
    如果存在常数$a>0,a_u>a_v>0$, 使得对任何$t\geq 0$和$u,v\in \realset^d$成立
    \begin{equation*}
        \left<u,f(t,u,v)\right>\leqslant a-a_u \|u\|^2+a_v \|v\|^2,
    \end{equation*}
    那么
    \begin{equation}\label{耗散性结果}
        \|x_n\|\leqslant \max\left(\left(a_u-a_v\right)^{-1/2}a^{1/2},\|x_0\|\right).
    \end{equation}
\end{theorem}
接下来讨论数值解的稳定性。为此,把\mainEquation 的初值条件改为$x(0)=y_0$, 用同样的数值算法(包括步长)产生数值解$\left(y_n\right)_{n=0}^\infty$和对“延迟”的近似$\left(\overline y_n\right)_{n=0}^\infty$, 并记$e_n:=y_n-x_n, \overline e_n:=\overline y_n-\overline x_n$. 接下来的\cref{stability}表明,在一定条件下,数值解稳定。
\begin{theorem}\label{stability}
    如果存在常数$b_u>b_v>0$, 使得对任何$t\geq 0$和$u,v,x,y\in \realset^d$成立
    \begin{equation*}
        \left\{
            \begin{aligned}
                &\left<f(t,u,v)-f(t,x,v),u-x\right>\leqslant -b_u \|u-x\|^2,
                \\ &\|f(t,u,v)-f(t,u,y)\|\leqslant b_v \|v-y\|,
            \end{aligned}
        \right.
    \end{equation*}
    那么
    \begin{equation*}
        \|e_n\|\leqslant \|e_0\|.
    \end{equation*}
\end{theorem}