这一节来叙述上一节中各定理的证明。
\begin{proof}[\cref{Peano}的证明]
    记$M<\infty$是集合$\left\{\left\|f(t,u,v)\right\|\colon t\in [0,1], u,v\in B_{2\|x_0\|}(0)\right\}$的一个上界,并取$0<h\leqslant 1$充分小,使得$\Gamma(\alpha + 1)^{-1} h^\alpha M \leqslant \|x_0\|$. 对于正整数$m$,作$t^m_{n}:=nh/m,\,n=0,1,2,\dots,m$, 然后按下式构造$\left(x^m_{n}\right)_{n=0}^m\subset \realset^d$,
    \begin{equation*}
        x^m_{n}=x_0+\Gamma(\alpha)^{-1}\sum_{k=1}^n\int_{t^m_{k-1}}^{t^m_k}\left(t^m_n-s\right)^{\alpha-1}f\bigl(s,x^m_{k-1},x^m_{q^m_{k-1}}\bigr)\differential s,\,n=1,2,\dots,m,
    \end{equation*}
    其中$q^m_{k}:=\lfloor qt^m_k \rfloor$. 利用这些有限长的序列,分段线性插值地构造连续函数$\left(x^m\right)_{m=1}^\infty\colon[0,h]\to \realset^d$,即
    \begin{equation*}
        x^m(t):=\frac{t^m_n-t}{t^m_n-t^m_{n-1}}x^m_{n-1}+\frac{t-t^m_{n-1}}{t^m_n-t^m_{n-1}}x^m_{n},\,t^m_{n-1}\leqslant t\leqslant t^m_n.
    \end{equation*}
    另外,为方便起见,对于$\delta>0$, 记$D(\delta):=\left\{(s,t)\in [0,h]\times [0,h]\colon 0\leqslant t-s<\delta\right\}$.

    现在证明$\|x^m(t^m_n)\|\leqslant 2\|x_0\|,\,0\leqslant t\leqslant h,n=0,1,2,\dots,m,m\in \positiveinteger$. 施归纳于$n$. $n=0$时显然。假设对于$0\leqslant k<n$成立$\|x^m(t^m_k)\|\leqslant 2\|x_0\|$, 根据$M$和$h$的定义,
    \begin{align*}
        \left\|x^m(t^m_n)\right\|&\leqslant \|x_0\|+\Gamma(\alpha)^{-1}\sum_{k=1}^{n}\int_{t^m_{k-1}}^{t^m_k} \left(t^m_n-s\right)^{\alpha-1} \bigl\|f\bigl(s,x^m_{k-1},x^m_{q^m_{k-1}}\bigr)\bigr\| \differential s
        \\ &\leqslant \|x_0\|+\Gamma(\alpha)^{-1} \sum_{k=1}^{n}\int_{t^m_{k-1}}^{t^m_k} \left(t^m_n-s\right)^{\alpha-1} M \differential s
        \\ &= \|x_0\|+\Gamma(\alpha)^{-1} M \int_{0}^{t^m_n} \left(t^m_n-s\right)^{\alpha-1} \differential s
        \\ &= \|x_0\|+\Gamma(\alpha+1)^{-1} M \left(t^m_n\right)^\alpha
        \\ &\leqslant \|x_0\|+\Gamma(\alpha+1)^{-1} M h^\alpha
        \leqslant 2\|x_0\|,
    \end{align*}
    这样就完成了归纳。而对任何$t\in [0,h]$, $t$必然落在某个形如$\left[t^m_{n-1},t^m_n\right]$的区间中,从而
    \begin{equation*}
        \left\|x^m(t)\right\|\leqslant \frac{t^m_n-t}{t^m_n-t^m_{n-1}}\left\|x^m_{n-1}\right\|+\frac{t-t^m_{n-1}}{t^m_n-t^m_{n-1}}\left\|x^m_{n}\right\|\leqslant 2\|x_0\|.
    \end{equation*}
    因此对每个$t\in [0,h]$都有$\left(x^m(t)\right)_{m=1}^\infty$在$\realset^d$中相对紧。

    然后讨论连续函数列$\left(x^m\right)_{m=1}^\infty$的等度连续性。首先,对于$0\leqslant k\leqslant n\leqslant m$, 有
    \begin{align*}
        &\relphantom{\leqslant}\Gamma(\alpha)\left\| x^m_n-x^m_k \right\|
        \\ &\leqslant \sum_{j=1}^{k}\int_{t^m_{j-1}}^{t^m_j}\left(\left(t^m_k-s\right)^{\alpha-1}-\left(t^m_n-s\right)^{\alpha-1}\right)M\differential s+\sum_{j=k+1}^{n}\int_{t^m_{j-1}}^{t^m_j}\left(t^m_n-s\right)^{\alpha-1}M\differential s
        \\ &=M\cdot\Bigl(\int_{0}^{t^m_k}\left(\left(t^m_k-s\right)^{\alpha-1}-\left(t^m_n-s\right)^{\alpha-1}\right)\differential s+\int_{t^m_k}^{t^m_n}\left(t^m_n-s\right)^{\alpha-1}\differential s\Bigr)
        \\ &=M\alpha^{-1}\cdot \left(\left(t^m_k\right) ^\alpha-\left(t^m_n\right)^\alpha+2\left(t^m_n-t^m_k\right)^\alpha\right)\leqslant 2M\alpha^{-1}\left(t^m_n-t^m_k\right)^\alpha.
    \end{align*}
    而至于$0\leqslant s\leqslant t\leqslant h$, 不妨设$s\in\left[t^m_{k-1},t^m_k\right],t\in\left[t^m_{n-1},t^m_n\right]$. 如果$k<n$, 那么
    \begin{align*}
        \left\| x^m(t)-x^m(s) \right\|
        &\leqslant \left\| x^m(t)-x^m(t^m_{n-1}) \right\|+\left\| x^m_{n-1}-x^m_k \right\|+\left\| x^m(t^m_k)-x^m(s) \right\|
        \\ &\leqslant \left\|x^m_{n}-x^m_{n-1}\right\|+\left\| x^m_{n-1}-x^m_k \right\|+\left\|x^m_{k}-x^m_{k-1}\right\|
        \\ &\leqslant 2M\Gamma(\alpha+1)^{-1}\left(\left(t^m_n-t^m_{n-1}\right)^\alpha + \left( t^m_{n-1}-t^m_{k}\right)^\alpha + \left(t^m_k-t^m_{k-1}\right)^\alpha\right)
        \\ &\leqslant 2M\Gamma(\alpha+1)^{-1}\left(2(h/m)^\alpha+(t-s)^\alpha\right),
    \end{align*}
    而若是$k=n$, 则有
    \begin{align*}
        \left\| x^m(t)-x^m(s) \right\| &\leqslant \left\|x^m_{n}-x^m_{n-1}\right\|
        \\ &\leqslant 2M\Gamma(\alpha+1)^{-1}\left(t^m_n-t^m_{n-1}\right)^\alpha
        \\ &\leqslant 2M\Gamma(\alpha+1)^{-1}(h/m)^\alpha,
    \end{align*}
    总之
    \begin{equation}\label{xmts}
        \left\| x^m(t)-x^m(s) \right\| \leqslant 2M\Gamma(\alpha+1)^{-1}\left(2(h/m)^\alpha+(t-s)^\alpha\right).
    \end{equation}
    任给$\varepsilon>0$, 取$N=N(\varepsilon)\in \positiveinteger$和$\delta_0=\delta_0(\varepsilon)>0$分别满足$2M\Gamma(\alpha+1)^{-1}\delta_0^\alpha < \varepsilon/2$和$4M\Gamma(\alpha+1)^{-1}(h/N)^\alpha<\varepsilon/2$. 由\cref{xmts}可知,$\left\| x^m(t)-x^m(s) \right\| < \varepsilon$对任何$m>N$及$(s,t)\in D(\delta_0)$成立。至于$1\leqslant m\leqslant N$, 由于每个$x^m$都是$[0,h]$上的一致连续函数,故存在有限个只依赖于$\varepsilon$的正数$\left(\delta_m\right)_{m=1}^N$, 使得对于$(s,t)\in D(\delta_m)$成立$\left\| x^m(t)-x^m(s) \right\| < \varepsilon$. 现在取$\delta=\delta(\varepsilon):=\min_{0\leqslant m\leqslant N} \delta_m>0$, 那么当$0\leqslant s\leqslant t \leqslant h$且$t-s<\delta$时,$\left\| x^m(t)-x^m(s) \right\| < \varepsilon$对于任何$m\in \positiveinteger$成立。这说明$\left(x^m\right)_{m=1}^\infty$是$C[0,h]$中的一致等度连续函数列。

    使用Arzel\`a-Ascoli定理,我们得到$\left\{x^m\colon m\in \positiveinteger\right\}$在$C\left([0,h],\realset^d\right)$中相对紧,故有一致收敛子列,这个子列仍记为$\left(x^m\right)_{m=1}^\infty$, 并设其极限函数为$x\in C\left([0,h],\realset^d\right)$. 任给$\varepsilon>0$, 由于$f$在紧集$[0,h]\times \overline{B_{2\|x_0\|}(0)}\times \overline{B_{2\|x_0\|}(0)}\subset [0,h]\times \realset^d\times \realset^d$上一致连续,故存在$\delta_1=\delta_1(\varepsilon)>0$使得$\|f(t,u,v)-f(t,x,y)\|<\varepsilon \alpha \Gamma(\alpha) h^{-\alpha} / 4$对$t\in [0,h]$以及满足$\|u-x\|+\|v-y\|<\delta_1$的$u,v,x,y\in B_{2\|x_0\|}(0)$成立。取$N_1=N_1(\delta_1)\in \positiveinteger$, 使得当$m>N_1$且$t\in [0,h]$时成立$\left\|x^m(t)-x(t)\right\|<\delta_1/2$, 从而
    \begin{equation}\label{xmx}
        \begin{aligned}
            &\relphantom{\leqslant}\left\|\int_{0}^{t}(t-s)^{\alpha-1}f(s,x^m(s),x^m(qs))\differential s - \int_{0}^{t}(t-s)^{\alpha-1}f(s,x(s),x(qs))\differential s\right\|
            \\ &\leqslant \int_{0}^{t}(t-s)^{\alpha-1} \frac{\varepsilon \alpha \Gamma(\alpha) h^{-\alpha}}{4} \differential s=\varepsilon \Gamma(\alpha) h^{-\alpha} t^\alpha / 4 \leqslant \varepsilon\Gamma(\alpha)/4.
        \end{aligned}
    \end{equation}

    由$\left(x^m\right)_{m=1}^\infty$的等度连续性,存在$\delta_2=\delta_2(\varepsilon)>0$使得$\left\|x^m(t)-x^m(s)\right\|<\min \left(\delta_1/2,\allowbreak\varepsilon/4\right)$对任何$m\in \positiveinteger$及$(s,t)\in D\left(\delta_2\right)$成立。取$N_2=N_2(\delta_2)\in \positiveinteger$使得$h/N_2<\delta_2$. 一方面,注意到$m>N_2$时总有$t^m_n-t^m_{n-1}<\delta_2<\delta_1/2,\,n=0,1,2,\dots,m$, 于是$\left\|f\bigl(t,x^m_{n-1},x^m_{q^m_{n-1}}\bigr)-f\bigl(t,x^m(t),x^m(qt)\bigr)\right\|<\varepsilon \alpha \Gamma(\alpha) h^{-\alpha} / 4,\,t\in\left[t^m_{n-1},t^m_n\right]$, 从而
    \begin{equation}\label{tn}
        \begin{aligned}
            &\relphantom{\leqslant}\Gamma(\alpha)\left\|x^m(t^m_n)-x_0-\Gamma(\alpha)^{-1}\int_{0}^{t^m_n}\left(t^m_n-s\right)^{\alpha-1}f(s,x^m(s),x^m(qs))\differential s\right\|
            \\ &\leqslant \sum_{k=1}^{n}\int_{t^m_{k-1}}^{t^m_k}\left(t^m_n-s\right)^{\alpha-1}\left\|f\bigl(s,x^m_{k-1},x^m_{q^m_{k-1}}\bigr)-f\bigl(s,x^m(s),x^m(qs)\bigr)\right\|\differential s
            \\ &\leqslant \sum_{k=1}^{n}\int_{t^m_{k-1}}^{t^m_k}\left(t^m_n-s\right)^{\alpha-1} \frac{\varepsilon \alpha \Gamma(\alpha) h^{-\alpha}}{4} \differential s
            \\ &=\int_{0}^{t^m_n} \left(t^m_n-s\right)^{\alpha-1} \frac{\varepsilon \alpha \Gamma(\alpha) h^{-\alpha}}{4} \differential s
            \leqslant \varepsilon\Gamma(\alpha)/4.
        \end{aligned}
    \end{equation}
    另一方面,任取$t\in [0,h]$, 设$t\in\left[t^m_{n-1},t^m_n\right]$, 则对任何$m\in \positiveinteger$成立
    \begin{equation}\label{xmtnt}
        \left\|x^m(t)-x^m(t^m_n)\right\|<\varepsilon/4.
    \end{equation}

    根据文献\inlinecite{Webb}中的性质3.2(3), 在$f$和$x$都连续时,$t\mapsto \Gamma(\alpha)^{-1}\int_{0}^{t}(t-s)^{\alpha-1}f(s,x(s),x(qs))\differential s$也是连续的, 因此可取$N_3=N_3(\varepsilon)\in \positiveinteger$充分大(从而$h/N_3>0$足够小),使得$m>N_3$且$(s,t)\in D(h/N_3)$时,$\Gamma(\alpha)^{-1}\allowbreak\left\|\int_{0}^{t}(t-\tau)^{\alpha-1}f(\tau,x(\tau),x(q\tau))\differential \tau\right.\allowbreak-\left.\int_{0}^{s}(s-\tau)^{\alpha-1}f(\tau,x(\tau),x(q\tau))\differential \tau\right\|\allowbreak <\varepsilon/4$. 于是当$t\in [0,h]$时,设$t\in\left[t^m_{n-1},t^m_n\right]$, 有
    \begin{equation}\label{xtnt}
        \begin{aligned}
            \Gamma(\alpha)^{-1}&\left\|\int_{0}^{t^m_n}(t^m_n-s)^{\alpha-1}f(s,x(s),x(qs))\differential s\right.
            \\&\phantom{\left\|\vphantom{\int_{0}^{t^m_n}}\right.}-\left.\int_{0}^{t}(t-s)^{\alpha-1}f(s,x(s),x(qs))\differential s\right\|<\varepsilon/4.
        \end{aligned}
    \end{equation}
    
    现在取$N=N(\varepsilon):=\max (N_1,N_2,N_3)$, 根据\cref{xmx,tn,xmtnt,xtnt}以及三角不等式,当$m>N,t\in [0,h]$时,设$t\in \left[t^m_{n-1},t^m_n\right]$, 那么可以估计
    \begin{align*}
        &\relphantom{\leqslant}\left\|x^m(t)-x_0-\Gamma(\alpha)^{-1}\int_{0}^{t}(t-s)^{\alpha-1}f(s,x(s),x(qs))\differential s\right\|
        \\ &\leqslant \left\|x^m(t)-x^m(t^m_n)\right\|
        \\ &\relphantom{\leqslant}+\left\|x^m(t^m_n)-x_0-\Gamma(\alpha)^{-1}\int_{0}^{t^m_n}(t^m_n-s)^{\alpha-1}f(s,x^m(s),x^m(qs))\differential s\right\|
        \\ &\relphantom{\leqslant}+\left\|x_0+\Gamma(\alpha)^{-1}\int_{0}^{t^m_n}(t^m_n-s)^{\alpha-1}f(s,x^m(s),x^m(qs))\differential s\right.
        \\ &\relphantom{\leqslant+} \phantom{\left\|\right.} \left.-x_0-\Gamma(\alpha)^{-1}\int_{0}^{t^m_n}(t^m_n-s)^{\alpha-1}f(s,x(s),x(qs))\differential s\right\|
        \\ &\relphantom{\leqslant}+\left\|x_0+\Gamma(\alpha)^{-1}\int_{0}^{t^m_n}(t^m_n-s)^{\alpha-1}f(s,x(s),x(qs))\differential s\right.
        \\ &\relphantom{\leqslant+} \phantom{\left\|\right.} \left.-x_0-\Gamma(\alpha)^{-1}\int_{0}^{t}(t-s)^{\alpha-1}f(s,x(s),x(qs))\differential s\right\|
        <\varepsilon,
    \end{align*}
    即$\lim_{m\to \infty}x^m(t)=x_0+\Gamma(\alpha)^{-1}\int_{0}^{t}(t-s)^{\alpha-1}f(s,x(s),x(qs))\differential s,\,\allowbreak 0\leqslant t\leqslant h$. 而极限的唯一性导致
    \begin{equation*}
        x(t)=\Gamma(\alpha)^{-1}\int_{0}^{t}(t-s)^{\alpha-1}f(s,x(s),x(qs))\differential s,\,0\leqslant t\leqslant h,
    \end{equation*}
    从而$x\in C([0,h],\realset^d)$是\mainEquation 在$[0,h]$上的一个弱解。
\end{proof}

\begin{proof}[\cref{Picard}的证明]
    \textbf{(存在性)} 构造Picard序列$\left(x_n\right)_{n=0}^\infty\colon [0,\infty)\to \realset^d$满足
    \begin{equation}\label{PicardSequence}
        \left\{\begin{aligned}
            x_{n+1}(t)&:=x_0 + \Gamma(\alpha)^{-1} \textstyle\int_0^t (t-s)^{\alpha-1} f(s,x_n(s),x_n(qs))\differential s, & n\in \naturalset,\\
            x_0(t)&:=x_0, & {}
        \end{aligned}\right.
    \end{equation}
    为方便,这里用$x_0\in C([0,\infty),\realset^d)$表达恒取$x_0\in\realset^d$的常值函数,通常不会引起混淆。记$M<\infty$是集合$\left\{\left\|f(t,u,v)\right\|\colon t\in [0,1], u,v\in B_{2\|x_0\|}(0)\right\}$的一个上界,并取$0<h\leqslant 1$充分小,使得$\Gamma(\alpha + 1)^{-1} h^\alpha M \leqslant \|x_0\|$. 可以归纳地证明$\left\|x_{n}(t)\right\|\leqslant 2\|x_0\|,\;t\in [0,h],n\in \naturalset$. 取$L:=L(2\|x_0\|)$, 那么对于$t\in [0,h],n\in \positiveinteger$,
    \begin{equation}\label{Picard估计递归}
        \begin{aligned}
            \left\| x_{n+1}(t)-x_n(t) \right\|&\leqslant L\Gamma(\alpha)^{-1}\int_0^t (t-s)^{\alpha-1} \\ &\relphantom{\leqslant}\bigl(\left\|x_n(s)-x_{n-1}(s)\right\| + \left\|x_n(qs)-x_{n-1}(qs)\right\|\bigr) \differential s.
        \end{aligned}
    \end{equation}

    现在归纳地说明
    \begin{equation}\label{Picard逐次估计}
        \left\| x_{n+1}(t) - x_n(t) \right\|\leqslant \frac{L^n M}{\Gamma(\alpha)^{n+1} \alpha} t^{(n+1)\alpha} \prod_{k=1}^n \left(1+q^{k\alpha}\right) \Beta(\alpha, k\alpha+1),\,t\in [0,h],n\in\naturalset.
    \end{equation}
    当$n=0$时,
    \begin{align*}
        \left\| x_1(t)-x_0(t) \right\|&=\| x_1(t)-x_0 \|\\
        &=\Gamma(\alpha)^{-1}\left\| \int_0^t (t-s)^{\alpha-1} f(s,x_0,x_0)\differential s\right\|\\
        &\leqslant \Gamma(\alpha)^{-1}\int_0^t (t-s)^{\alpha-1} M\differential s\\
        &=\Gamma(\alpha)^{-1} \alpha^{-1} t^\alpha M.
    \end{align*}
    假定\cref{Picard逐次估计}在$n$取$n-1$时成立,然后
    \begin{align*}
        &\relphantom{\leqslant} \left\| x_{n+1}(t)-x_n(t) \right\| \\ &\leqslant \frac{L}{\Gamma(\alpha)}\int_0^t (t-s)^{\alpha-1} \bigl(\left\|x_n(s)-x_{n-1}(s)\right\| + \left\|x_n(qs)-x_{n-1}(qs)\right\|\bigr) \differential s\\
        &\leqslant \frac{L^{n} M}{\Gamma(\alpha)^{n+1} \alpha} \Bigl(\prod_{k=1}^{n-1}\left(1+q^{k\alpha}\right) \Beta(\alpha, k\alpha+1)\Bigr)\int_0^t (t-s)^{\alpha-1} \bigl(s^{n\alpha} + (qs)^{n\alpha}\bigr) \differential s\\
        &=\frac{L^{n} M}{\Gamma(\alpha)^{n+1} \alpha} \Bigl(\prod_{k=1}^{n-1}\left(1+q^{k\alpha}\right) \Beta(\alpha, k\alpha+1)\Bigr) \left(1+q^{n\alpha}\right) \int_0^t (t-s)^{\alpha-1} s^{n\alpha} \differential s\\
        &=\frac{L^{n} M}{\Gamma(\alpha)^{n+1} \alpha} \Bigl(\prod_{k=1}^{n-1}\left(1+q^{k\alpha}\right) \Beta(\alpha, k\alpha+1)\Bigr) \left(1+q^{n\alpha}\right) t^{n\alpha+\alpha}\Beta(\alpha, n\alpha+1)\\
        &=\frac{L^n M}{\Gamma(\alpha)^{n+1} \alpha} t^{(n+1)\alpha} \prod_{k=1}^n \left(1+q^{k\alpha}\right) \Beta(\alpha, k\alpha+1).
    \end{align*}
    由归纳原理,\cref{Picard逐次估计}成立。进一步地,注意到
    \begin{equation*}
        \prod_{k=1}^n \left(1+q^{k\alpha}\right)\leqslant \prod_{k=1}^n \exp\left(q^{k\alpha}\right)=\exp \sum_{k=1}^n \left(q^\alpha\right)^k\leqslant \exp \frac{q^\alpha}{1-q^\alpha}
    \end{equation*}
    和
    \begin{equation*}
        \prod_{k=1}^n \Beta(\alpha, k\alpha+1)=\prod_{k=1}^n \frac{\Gamma(\alpha) \Gamma(k\alpha+1)}{\Gamma((k+1)\alpha+1)}=\Gamma(\alpha)^n \frac{\Gamma(\alpha+1)}{\Gamma((n+1)\alpha+1)},
    \end{equation*}
    我们有
    \begin{equation*}
        \left\| x_{n+1}(t) - x_n(t) \right\|\leqslant \frac{L^n M t^{(n+1)\alpha}}{\Gamma((n+1)\alpha+1)} \exp \frac{q^\alpha}{1-q^\alpha},\,t\in [0,h].
    \end{equation*}

    由Cauchy-Hadamard公式和Stirling公式易知Mittag-Leffler函数\cite{book}$E_\alpha(z)=\sum_{n=0}^{\infty}\Gamma(\alpha n+1)^{-1}z^n$对于任何$z\in \complexset$收敛,然后根据Weierstrass M判别法就得到函数项级数$\sum_{n=0}^\infty \left(x_{n+1}-x_n\right)$在$[0,h]$上绝对一致收敛,于是函数列$\left(x_n\right)_{n=0}^\infty$在$[0,h]$上存在一致极限$x$. 这说明任取$\varepsilon>0$, 总存在$N=N(\varepsilon)\in \naturalset$, 使得$\left\|x_n(t)-x(t)\right\|<\varepsilon$在$n>N$且$t\in [0,h]$时成立。这样一来,当$t\in [0,h]$时,
    \begin{align*}
        &\relphantom{\leqslant} \left\| \int_0^t (t-s)^{\alpha-1} f(s,x_n(s),x_n(qs))\differential s - \int_0^t (t-s)^{\alpha-1} f(s,x(s),x(qs))\differential s \right\|\\
        &\leqslant \int_0^t (t-s)^{\alpha-1} L\cdot\left(\left\|x_n(s)-x(s)\right\|+\left\|x_n(qs)-x(qs)\right\|\right) \differential s\\
        &\leqslant 2\varepsilon L\int_0^t (t-s)^{\alpha-1}\differential s = 2\varepsilon L\alpha^{-1}t^\alpha,
    \end{align*}
    因此
    \begin{equation*}
        \lim_{n\to\infty} \int_0^t (t-s)^{\alpha-1} f(s,x_n(s),x_n(qs))\differential s = \int_0^t (t-s)^{\alpha-1} f(s,x(s),x(qs))\differential s.
    \end{equation*}
    现在在\cref{PicardSequence}中命$n\to\infty$就得到
    \begin{equation*}
        x(t)=x_0 + \Gamma(\alpha)^{-1}\int_0^t (t-s)^{\alpha-1} f(s,x(s),x(qs))\differential s,\;0\leqslant t\leqslant h.
    \end{equation*}
    利用文献\inlinecite{Webb}中的性质3.2(3)(6), 容易归纳地得到$\left(x_n\right)_{n=0}^\infty \subseteq C^{0,\alpha} \cap AC [0,h] \subseteq C[0,h]$, 于是其一致极限$x\in C[0,h]$. 这样$x$就是\mainEquation 在$[0,h]$上的一个弱解。

    如果$L$可以不依赖于$r$, 那么\cref{Picard估计递归}对于一切$t\in [0,\infty)$成立,由此可见\mainEquation 的弱解将在$[0,\infty)$上全局存在。

    \textbf{(唯一性)} 先设$0<T:=\sup I<\infty$, $[0,T]$中的$\realset^d$值连续函数$x,y$都是\mainEquation 的弱解。记$L:=L\left(\max\left(\max_{0\leqslant t\leqslant T}\|x(t)\|,\right.\right.\allowbreak\left.\left.\max_{0\leqslant t\leqslant T}\|y(t)\|\right)\right)$, 并作$S:=\{t\in [0,T]\colon x(t)\neq y(t)\},t_*:=\inf S$, 下证$t_*=\infty$. 反证,假设$0\leqslant t_*\leqslant T$, 分3种情况讨论。
    \begin{enumerate}
        \item 如果$t_*=T$, 此时必有$T\in I$, 且在$[0,T)$上$x=y$, 而$x,y$都是连续的,因此必在闭区间$[0,T]$上处处相等,此时$S=\varnothing,t_*=\infty$, 矛盾。
        \item 如果$0<t_*<T$, 那么在$[0,t_*)$上有$x=y$. 选取$\delta>0$充分小,使得$t_*+\delta\leqslant T$且$q\cdot (t_*+\delta)<t_*$, 然后就有$x(qt)=y(qt),\,0\leqslant t\leqslant t_*+\delta$. 当$t\in \left[0,t_*+\delta\right]$时,
        \begin{align*}
            \left\| x(t) - y(t) \right\| &\leqslant {L}{\Gamma(\alpha)}^{-1} \int_0^t (t-s)^{\alpha-1} \bigl(\|x(s)-y(s)\|+\|x(qs)-y(qs)\|\bigr) \differential s\\
            &={L}{\Gamma(\alpha)}^{-1} \int_0^t (t-s)^{\alpha-1} \|x(s)-y(s)\| \differential s,
        \end{align*}
        然后利用分数阶Gronwall不等式\cite{mild}就得到在$\left[0,t_*+\delta\right]$上都有$x=y$, 故$t_*\geqslant t_*+\delta$, 而这是不可能的。
        \item 如果$t_*=0$, 选取$\delta\in (0,T]$充分小,使得$2L\Gamma(\alpha+1)^{-1}\delta^\alpha<1$. 当$t\in[0,\delta]$时,
        \begin{align*}
            \left\| x(t) - y(t) \right\| &\leqslant L\Gamma(\alpha)^{-1}\int_0^t (t-s)^{\alpha-1} \bigl(\|x(s)-y(s)\|+\|x(qs)-y(qs)\|\bigr) \differential s\\
            &\leqslant 2L\Gamma(\alpha)^{-1} \Bigl(\max_{0\leqslant s\leqslant t}\|x(s)-y(s)\|\Bigr) \int_0^t (t-s)^{\alpha-1} \differential s\\
            &=2L\Gamma(\alpha+1)^{-1}t^\alpha \max_{0\leqslant s\leqslant t}\|x(s)-y(s)\|.
        \end{align*}
        上式两边对$t\in[0,\delta]$取最大值,得到
        \begin{equation*}
            \max_{0\leqslant t\leqslant \delta}\|x(t)-y(t)\|\leqslant 2L\Gamma(\alpha+1)^{-1}\delta^\alpha \max_{0\leqslant t\leqslant \delta}\|x(t)-y(t)\|,
        \end{equation*}
        结合$\delta$的选取知道只可能有$\max_{0\leqslant s\leqslant \delta}\|x(s)-y(s)\|=0$, 即等式$x=y$至少在$[0,\delta]$上成立, 故$t_*\geqslant \delta$, 矛盾.
    \end{enumerate}
    综合以上各种情况知$t_*\notin [0,T]$, 这只可能是$t_*=\infty$, 此时必有$S=\varnothing$, 故而$x$和$y$在整个$I$上相等。而如若$x,y$是$[0,\infty)$上\mainEquation 的弱解,上述结果则表明它们在任何有限区间$[0,T]$上相等,因而在$[0,\infty)$上相等。唯一性证毕。
\end{proof}

\begin{proof}[\cref{dissipativity}的证明]
    \cref{L1格式}两边与$x_n$作内积并结合Cauchy-Schwarz不等式得到
    \begin{equation}\label{耗散性-内积放缩}
        \begin{aligned}
            \|x_n\|^2&\leqslant \left(t_n-t_{n-1}\right)^\alpha \Bigl( \Gamma(2-\alpha) \left(a-a_u \|x_n\|^2 + a_v \|\overline x_n\|^2\right)
            \\ &\relphantom{\leqslant} + \sum_{k=0}^{n-1} \left(a_{n,k+1}-a_{n,k}\right) \frac{\|x_k\|^2+\|x_n\|^2}{2}\Bigr).
        \end{aligned}
    \end{equation}
    注意到$\sum_{k=0}^{n-1} \left(a_{n,k+1}-a_{n,k}\right)=a_{n,n}=\left(t_n-t_{n-1}\right)^{-\alpha}$, 有
    \begin{equation}\label{耗散性-求和}
        \left(t_n-t_{n-1}\right)^\alpha \sum_{k=0}^{n-1} \left(a_{n,k+1}-a_{n,k}\right) \frac{\|x_k\|^2+\|x_n\|^2}{2}\leqslant\frac{1}{2}\Bigl(\max_{0\leqslant k<n} \|x_k\|^2 + \|x_n\|^2\Bigr).
    \end{equation}
    将\cref{耗散性-求和}代入\cref{耗散性-内积放缩}得到
    \begin{align*}
        \|x_n\|^2 &\leqslant a_n \left(a-a_u \|x_n\|^2 + a_v \|\overline x_n\|^2\right) + \max_{0\leqslant k<n} \|x_k\|^2
        \\ &\leqslant a_n \Bigl(a-a_u \|x_n\|^2 + a_v \max_{0\leqslant k\leqslant n} \|x_k\|^2\Bigr) + \max_{0\leqslant k<n} \|x_k\|^2,
    \end{align*}
    其中$a_n:=2\left(t_n-t_{n-1}\right)^\alpha \Gamma(2-\alpha)$. 如果$\|x_n\|=\max_{0\leqslant k\leqslant n}\|x_k\|$, 那么$\|x_n\|^2 \leqslant a_n \left(a-a_u \|x_n\|^2 + a_v \|x_n\|^2\right) + \|x_n\|^2$, 此时$\|x_n\|^2 \leqslant (a_u-a_v)^{-1}a$; 否则$\|x_n\|<\max_{0\leqslant k<n}\|x_k\|$.
    因此,无论如何总有
    \begin{equation*}
        \|x_n\|^2\leqslant \max\bigl((a_u-a_v)^{-1}a,\max_{0\leqslant k<n}\|x_k\|^2\bigr),\,n\in\positiveinteger.
    \end{equation*}
    最后只需归纳即可得到想要的结论。
\end{proof}

\begin{proof}[\cref{stability}的证明]
    写出$\left(e_n\right)_{n=0}^\infty$满足的等式为
    \begin{equation*}
        e_n=\left(t_n-t_{n-1}\right)^\alpha\Bigl(\sum_{k=0}^{n-1}\left(a_{n,k+1}-a_{n,k}\right)e_k+\Gamma(2-\alpha)\left(f\left(t_n,y_n,\overline y_n\right)-f\left(t_n,x_n,\overline x_n\right)\right)\Bigr),
    \end{equation*}
    两边同$e_n$取内积,得
    \begin{align*}
        \|e_n\|^2&\leqslant\left(t_n-t_{n-1}\right)^\alpha\Bigl(\sum_{k=0}^{n-1}\left(a_{n,k+1}-a_{n,k}\right)\frac{\|e_k\|^2+\|e_n\|^2}{2}+\Gamma(2-\alpha)
        \\& \relphantom{=}\left(\left<f\left(t_n,y_n,\overline y_n\right)-f\left(t_n,x_n,\overline y_n\right),e_n\right>+\left\|f\left(t_n,x_n,\overline y_n\right)-f\left(t_n,x_n,\overline x_n\right)\right\|\left\|e_n\right\|\right)\Bigr)
        \\&\leqslant 2^{-1}\Bigl(\max_{0\leqslant k<n} \|e_k\|^2 + \|e_n\|^2\Bigr)
        +2^{-1}a_n\left(-b_u \|e_n\|^2+b_v \|\overline e_n\|^2\right),
    \end{align*}
    其中$a_n:=2\left(t_n-t_{n-1}\right)^\alpha\Gamma(2-\alpha)$. 注意到$\|\overline e_n\|\leqslant \max_{0\leqslant k\leqslant n} \|e_k\|$, 我们有
    \begin{equation*}
        \|e_n\|^2\leqslant \max_{0\leqslant k<n} \|e_k\|^2+a_n\Bigl(-b_u \|e_n\|^2+b_v \max_{0\leqslant k\leqslant n} \|e_k\|^2\Bigr).
    \end{equation*}
    一种情况是$\|e_n\|=\max_{0\leqslant k\leqslant n} \|e_k\|$, 此时$\max_{0\leqslant k<n} \|e_k\|^2 \geqslant \|e_n\|^2 (1+ a_n(b_u-b_v))\geqslant \|e_n\|^2=\max_{0\leqslant k\leqslant n} \|e_k\|^2\geqslant \max_{0\leqslant k<n} \|e_k\|^2$, 这个不等式链的最左端和最右端一样,因此其中的不等号全取等,特别地有$\|e_n\|=\max_{0\leqslant k<n} \|e_k\|$. 而另一种情况是$\|e_n\|<\max_{0\leqslant k\leqslant n} \|e_k\|$, 此时显然有$\|e_n\|<\max_{0\leqslant k< n} \|e_k\|$. 总之,
    \begin{equation*}
        \|e_n\|\leqslant\max_{0\leqslant k< n} \|e_k\|,\,n\in\positiveinteger.
    \end{equation*}
    最后归纳即得结论。
\end{proof}