\AddToHook{package/xeCJK/after}{\defaultCJKfontfeatures{}}
% \PassOptionsToPackage{quiet}{xeCJK}

\documentclass[aspectratio=16 9, 10pt, notheorems]{ctexbeamer}
\usepackage[sort&compress, capitalise]{cleveref}
\let\oldcref\cref
\renewcommand{\cref}[1]{\oldcref{#1}~}
\crefname{theorem}{定理}{定理}
\crefname{equation}{式}{式}
\newcommand\crefpairconjunction{~和~}
\newcommand\crefrangeconjunction{--}
\let\oldeqref\eqref
\renewcommand{\eqref}[1]{~\oldeqref{#1}~}
\newcommand{\mainEquation}{方程\eqref{方程}}

\usefonttheme{professionalfonts}
\usepackage{setspace}
% \usepackage[space, hyperref]{ctex}
\usepackage{enumitem}
% \setbeamertemplate{itemize items}[circle]

\usepackage{amsthm, amsmath, amssymb}
\allowdisplaybreaks
\newcommand{\differential}{\mathop{}\!\mathrm{d}}
\newcommand{\relphantom}[1]{\mathrel{\phantom{#1}}\negmedspace {}}
\newcommand{\realset}{\mathbb{R}}
\newcommand{\RiemannLiouvilleIntegral}{I^\alpha_{0^+}}
\newcommand{\Caputo}{{^C}{D}^\alpha_{0^+}}
\newcommand{\naturalset}{\mathbb{N}}
\newcommand{\complexset}{\mathbb{C}}
\newcommand{\positiveinteger}{\mathbb{N_+}}
\newcommand{\Beta}{\mathrm{B}}

\setbeamertemplate{navigation symbols}{}

\usepackage{xcolor}
\definecolor{structurecolor}{RGB}{20,65,137}
\setbeamercolor{structure}{fg=structurecolor}
\setbeamercolor{titlelike}{fg=white}
\setbeamercolor{informationlike}{}
\setbeamertemplate{background canvas}{\includegraphics[page=3, height=\paperheight]{template.pdf}}
\setbeamertemplate{title page}% 定义标题页模板
{
    % \vfill%竖直方向空白填充,和下面的\vifll结合使用可以使它们之间的内容竖直居中
    \vskip2.3cm
    \begin{beamercolorbox}[center]{titlelike}%内容居中对齐,边界拐角使用圆角
        \usebeamerfont{title}\inserttitle\par%%插入标题内容,并对字体进行了设置
        \ifx\insertsubtitle\@empty%%\ifx的作用是比较两个字符串是否相等,这里的作用是判断是否有子标题,如果没有,就什么也不做
        \else%%有子标题
        	\vskip0.25em%%竖直跳过一段距离
        	{\usebeamerfont{subtitle}\usebeamercolor[fg]{subtitle}\insertsubtitle\par}%%插入子标题,并设置了所使用的字体和前景颜色
        \fi%
    \end{beamercolorbox}%
    \vskip2cm\par%%竖直跳过一段距离
    \begin{beamercolorbox}[center]{informationlike}%内容居中对齐
        \usebeamerfont{author}\insertauthor\\
        \vspace{.8em}
        \usebeamerfont{institute}\insertinstitute\\
        \vspace{.8em}
        \usebeamerfont{date}\insertdate
    \end{beamercolorbox}
    % \vfill
}
\setbeamertemplate{frametitle}
{
    \begin{beamercolorbox}[wd=\paperwidth,ht=5ex,dp=1.125ex,left]{titlelike}
        \hspace{2em}\usebeamerfont{frametitle}\insertframetitle
    \end{beamercolorbox}
}
\setbeamertemplate{section in toc}[ball unnumbered]% 正方形标号square, 圆形标号circle
\setbeamertemplate{subsection in toc}[square]% 正方形标号square
% \setbeamertemplate{content}
% {
%     \begin{beamercolorbox}[wd=\paperwidth,ht=5.5ex,dp=1.125ex,left]{titlelike}
%         \hspace{2em}\usebeamerfont{frametitle}\insertframetitle
%     \end{beamercolorbox}
% }
% \setbeamerfont{section in toc}{size=\Huge}
% \setbeamerfont{section in toc shaded}{size=\normalsize}
% \usepackage{xpatch}
% \makeatletter
% \beamer@tocsectionnumber=-2147483647\relax
% \xpatchcmd\beamer@sectionintoc
%   {\ifnum\beamer@tempcount>0}
%   {\ifnum\beamer@tempcount>-1}
%   {}{\fail}
% \makeatother
\definecolor{tocshaded}{RGB}{166,166,166}
\setbeamercolor{section in toc shaded}{fg=tocshaded}
\setbeamercolor{subsection in toc shaded}{fg=tocshaded}
\setbeamertemplate{section in toc shaded}[default][100]
\setbeamertemplate{subsection in toc shaded}[default][100]
\AtBeginSection[]
{
    {
        \setbeamertemplate{background canvas}{\includegraphics[page=4, height=\paperheight]{template.pdf}}
        \begin{frame}
            % \transfade%淡入淡出
            \tableofcontents[currentsection]
            \vspace{5ex}
        \end{frame}
    }
}

% \theoremstyle{plain}
% \renewcommand{thetheorem}{\arabic{section}.\arabic{theorem}}
% \setbeamercolor{block}{fg=white}
\setbeamertemplate{theorems}[numbered]
% \setbeamercolor{block title}{bg=structure, fg=white}
\setbeamercolor{block title}{bg=structure!30}
\setbeamercolor{block body}{bg=structure!20}
\newtheorem*{definition}{定义}
\newtheorem{theorem}{定理}
\newtheorem{lemma}{引理}
\newtheorem*{property}{性质}

% \newcounter{definition}%[section]
% \renewcommand{\thedefinition}{\textbf{\arabic{definition}.}}
% \newenvironment{definition}{\addtocounter{definition}{1}定义}{\\}

% \newcounter{theorem}
% \renewcommand{\thetheorem}{\textbf{\arabic{theorem}.}}
% \newenvironment{theorem}{\addtocounter{theorem}{1}\\定理\thetheorem}{\\}

% \newcounter{lemma}
% \renewcommand{\thelemma}{\textbf{\arabic{lemma}.}}
% \newenvironment{lemma}{\addtocounter{lemma}{1}引理\thelemma}{\\}

% \newcounter{property}
% \renewcommand{\theproperty}{\textbf{\arabic{property}.}}
% \newenvironment{property}{\addtocounter{property}{1}\\性质\theproperty}{\\}

\setbeamertemplate{frametitle continuation}{}
\title{分数阶比例延迟方程的几种数值方法的研究}
\author{汇报人:李云鹏}
\institute{导\phantom{字}师:雷\phantom{字}强}
\date{2024年6月28日}
\setbeamerfont{institute}{size=\fontsize{10}{0}}
\setbeamertemplate{bibliography item}[text]

\begin{document}
{
    \setbeamertemplate{background canvas}{\includegraphics[page=1, height=\paperheight]{template.pdf}}
    \begin{frame}\titlepage\end{frame}
}

{
% \titlecontents{section}[10em]{\bfseries \zihao{5} \vspace{10pt}}{\contentslabel{4em}}{\hspace*{-4em}}{~\titlerule*[0.6pc]{$.$}~\contentspage}
    \setbeamertemplate{background canvas}{\includegraphics[page=2, height=\paperheight]{template.pdf}}
    \begin{frame}
        % \vfill
        % \begin{beamercolorbox}[wd=\paperwidth,ht=7.5ex,dp=1.125ex,left]{}%
        %     \hspace{10em}
        %     \tableofcontents%
        % \end{beamercolorbox}
        \begin{columns}
            \column{.4\textwidth}
            \column{.6\textwidth}
            % \titlespacing{\section}{\parindent}{2}{\wordsep}
            \vfill
            \begin{spacing}{1.4}
                \tableofcontents[hideallsubsections]
            \end{spacing}
            \vfill
        \end{columns}
        % \vfill
    \end{frame}
}

\section{课题主要研究内容}
\begin{frame}{课题主要研究内容}
    设$d\in \positiveinteger$. 本课题主要研究如下的分数阶比例延迟方程,
\begin{equation}\label{方程}
    \begin{cases}%dcases
        \Caputo x(t) = f(t, x(t), x(qt)),&t\geqslant 0,\\
        x(0) = x_0,
    \end{cases}
\end{equation}
其中$x$是$\realset^d$值未知函数,$x_0\in \realset^d$给定,$0<\alpha<1$, 函数$f:[0,\infty)\times \realset^d\times \realset^d\to\realset^d$连续,$\Caputo x(t)=\Gamma(1-\alpha)^{-1}\int_{0}^{t}(t-s)^{-\alpha}x'(s)\differential s$.

\begin{definition}
    如果有函数$x$, 使得$t$属于$[0,\infty)$的某个包含$0$的子区间$I$上时满足
\begin{equation*}\label{mild solution}
    x(t)=x_0+\Gamma(\alpha)^{-1}\int_{0}^{t} (t-s)^{\alpha-1} f(s,x(s),x(qs)) \differential s,
\end{equation*}
那么称$x$是\mainEquation 在$I$上的一个弱解。
\end{definition}
\end{frame}

\section{目前已完成的研究工作及结果}
\begin{frame}{弱解的存在性与唯一性}
    \begin{theorem}[Peano存在性定理]
        \mainEquation 总是在某个小区间$[0,h]$上存在弱解。
    \end{theorem}
    \begin{theorem}[Picard存在唯一性定理]
        如果$f(t,\cdot,\cdot)$对$t\in [0,\infty)$一致地局部Lipschitz, 即对任何$r>0$, 存在不依赖于$t$的$L=L(r)\geqslant 0$, 使得
        \begin{equation*}\label{Lipschitz}
            \| f(t,x,y) - f(t,u,v) \| \leqslant L\cdot (\|x-u\| + \|y-v\|)
        \end{equation*}
        对任何$t\in [0,\infty)$以及$x,y,u,v\in B_r(0)$成立,那么\mainEquation 在某个小区间$[0,h]$上存在弱解,并且弱解在存在区间$I\ni 0$上唯一。进一步地,如果$L$可以不依赖于$r$, 那么在$[0,\infty)$上全局存在唯一的弱解。
    \end{theorem}
\end{frame}
\begin{frame}{Peano存在性定理的证明概要}
    取时间序列$t^m_{n}:=nh/m,\,n=0,1,2,\dots,m$, 并构造Euler折线$\left(x^m\right)_{m=1}^\infty\colon[0,h]\to \realset^d$如下。
    \begin{equation*}
        x^m_{n}=x_0+\Gamma(\alpha)^{-1}\sum_{k=1}^n\int_{t^m_{k-1}}^{t^m_k}\left(t^m_n-s\right)^{\alpha-1}f\bigl(s,x^m_{k-1},x^m_{q^m_{k-1}}\bigr)\differential s,\,n=1,2,\dots,m,
    \end{equation*}
    \begin{equation*}
        x^m(t):=\frac{t^m_n-t}{t^m_n-t^m_{n-1}}x^m_{n-1}+\frac{t-t^m_{n-1}}{t^m_n-t^m_{n-1}}x^m_{n},\,t^m_{n-1}\leqslant t\leqslant t^m_n,
    \end{equation*}
    其中$h>0$取充分小, 使得Euler折线一致有界。
\end{frame}
\begin{frame}{Peano存在性定理的证明概要}
    可以估计出
    \begin{equation*}\label{xmts}
        \left\| x^m(t)-x^m(s) \right\| \leqslant 2M\Gamma(\alpha+1)^{-1}\left(2(h/m)^\alpha+(t-s)^\alpha\right),
    \end{equation*}
    进而知道$\left(x^m\right)_{m=1}^\infty$等度连续。
    \\~\\
    使用Arzel\`a-Ascoli定理,$\left(x^m\right)_{m=1}^\infty$有一致收敛子列,仍记为$\left(x^m\right)_{m=1}^\infty$, 并设其极限函数为$x\in C\left([0,h],\realset^d\right)$.
    \\~\\
    接下来证明$x$是\mainEquation 在$[0,h]$上的一个弱解。
\end{frame}
\begin{frame}[allowframebreaks]{Peano存在性定理的证明概要}
    借助三角不等式,对任何$t\in [0,h]$, 设$t\in\left[t^m_{n-1},t^m_n\right]$, 有
    \begin{align*}
        &\relphantom{\leqslant}\left\|x^m(t)-x_0-\Gamma(\alpha)^{-1}\int_{0}^{t}(t-s)^{\alpha-1}f(s,x(s),x(qs))\differential s\right\|
        \\ &\leqslant \left\|x^m(t)-x^m(t^m_n)\right\|
        \\ &\relphantom{\leqslant}+\left\|x^m(t^m_n)-x_0-\Gamma(\alpha)^{-1}\int_{0}^{t^m_n}(t^m_n-s)^{\alpha-1}f(s,x^m(s),x^m(qs))\differential s\right\|
        \\ &\relphantom{\leqslant}+\Gamma(\alpha)^{-1}\left\|\int_{0}^{t^m_n}(t^m_n-s)^{\alpha-1}\left(f(s,x^m(s),x^m(qs))-f(s,x(s),x(qs))\right)\differential s\right\|
        \\ &\relphantom{\leqslant}+\Gamma(\alpha)^{-1}\left\|\int_{0}^{t^m_n}(t^m_n-s)^{\alpha-1}f(s,x(s),x(qs))\differential s-\int_{0}^{t}(t-s)^{\alpha-1}f(s,x(s),x(qs))\differential s\right\|.
    \end{align*}
\end{frame}
\begin{frame}{Peano存在性定理的证明概要}
    可以证明,当$m$充分大时,不等号右边的每一项都可以任意小,从而
    \begin{equation*}
        x(t)=x_0+\Gamma(\alpha)^{-1}\int_{0}^{t}(t-s)^{\alpha-1}f(s,x(s),x(qs))\differential s,\,0\leqslant t\leqslant h.
    \end{equation*}
    从而$x\in C([0,h],\realset^d)$是\mainEquation 在$[0,h]$上的一个弱解。
\end{frame}

\begin{frame}{Picard存在性定理的证明概要}
    构造Picard序列$\left(x_n\right)_{n=0}^\infty\colon [0,\infty)\to \realset^d$满足
    \begin{equation*}\label{PicardSequence}
        \left\{\begin{aligned}
            x_{n+1}(t)&:=x_0 + \Gamma(\alpha)^{-1} \textstyle\int_0^t (t-s)^{\alpha-1} f(s,x_n(s),x_n(qs))\differential s, & n\in \naturalset,\\
            x_0(t)&:=x_0. & {}
        \end{aligned}\right.
    \end{equation*}
    取$h$充分小,使得$\left(x_n\right)_{n=0}^\infty$一致有界。(这是为了在证明过程中取固定的Lipschitz常数。如果$L$可以与$r$无关,那么就可以在整个$[0,\infty)$上运作下面的证明。)归纳知
    \begin{equation*}
        \left\| x_{n+1}(t) - x_n(t) \right\|\leqslant \frac{L^n M t^{(n+1)\alpha}}{\Gamma((n+1)\alpha+1)} \exp \frac{q^\alpha}{1-q^\alpha},\,t\in [0,h].
    \end{equation*}
    由Mittag-Leffler函数的性质知$\left(x_n\right)_{n=0}^\infty$在$[0,h]$上一致收敛,然后容易看出其极限函数即是\mainEquation 在$[0,h]$上的一个弱解。
\end{frame}
\begin{frame}[allowframebreaks]{Picard唯一性定理的证明概要}
    先设$0<T:=\sup I<\infty$, $[0,T]$中的$\realset^d$值连续函数$x,y$都是\mainEquation 的弱解。记$L:=L\left(\max\left(\max_{0\leqslant t\leqslant T}\|x(t)\|,\right.\right.\allowbreak\left.\left.\max_{0\leqslant t\leqslant T}\|y(t)\|\right)\right)$, $S:=\{t\in \colon x(t)\neq y(t)\},t_*:=\inf S$, \\下证$t_*=\infty$. 反证,假设$0\leqslant t_*\leqslant T$.
    \\~\\
    \begin{itemize}%[ball unnumbered]
        % \setbeamertemplate{itemize items}{$\bullet$}
        \item[$\bullet$] 如果$t_*=T$, 那么$x(T)=y(T)$, 此时容易看出矛盾。
        \item[$\bullet$] 如果$0<t_*<T$, 那么对于充分小的$\delta>0$, 当$t\in \left[0,t_*+\delta\right]$时有$x(qs)=y(qs)$, 从而
        \begin{equation*}
            \left\| x(t) - y(t) \right\| \leqslant {L}{\Gamma(\alpha)}^{-1} \int_0^t (t-s)^{\alpha-1} \|x(s)-y(s)\| \differential s,
        \end{equation*}
        然后利用分数阶Gronwall不等式就推出矛盾。
        \item[$\bullet$] 如果$t_*=0$, 那么对于充分小的$\delta>0$, 当$t\in \left[0,\delta\right]$时,
        \begin{align*}
            \left\| x(t) - y(t) \right\| \leqslant 2L\Gamma(\alpha+1)^{-1}t^\alpha \max_{0\leqslant s\leqslant t}\|x(s)-y(s)\|,
        \end{align*}
        两边对$t\in[0,\delta]$取最大值,得到
        \begin{equation*}
            \max_{0\leqslant t\leqslant \delta}\|x(t)-y(t)\|\leqslant 2L\Gamma(\alpha+1)^{-1}\delta^\alpha \max_{0\leqslant t\leqslant \delta}\|x(t)-y(t)\|,
        \end{equation*}
        因此只要选取$2L\Gamma(\alpha+1)^{-1}\delta^\alpha<1$就能看出在$[0,\delta]$上成立$x=y$, 这也导致矛盾。
    \end{itemize}
    综合以上各种情况知$x$和$y$在整个$I$上相等。
    \\~\\如若$x,y$是$[0,\infty)$上\mainEquation 的弱解,上述结果表明它们在任何有限区间$[0,T]$上相等,因而在$[0,\infty)$上相等。唯一性证毕。
\end{frame}

\begin{frame}{L1格式}
    在叙述数值解部分的结果之前,先给出数值格式。取严格递增趋于正无穷的序列$\left(t_n\right)_{n=0}^\infty$作为时间节点,其中$t_0=0$. 利用在每个小区间上线性插值的办法来近似导数和延迟,就得到针对\mainEquation 的L1数值格式
    \begin{equation*}\label{L1格式}
        x_n=\left(t_n-t_{n-1}\right)^\alpha \Bigl(\Gamma(2-\alpha) f\left(t_n,x_n,\overline{x}^n\right) + \sum_{k=0}^{n-1} \left(a_{n,k+1}-a_{n,k}\right) x_k\Bigr),
    \end{equation*}
    其中$a_{n,k}:=\frac{\left(t_n-t_{k-1}\right)^{1-\alpha}-\left(t_n-t_{k}\right)^{1-\alpha}}{t_k-t_{k-1}} (1\leqslant k\leqslant n),a_{n,0}:=0,\overline x_n$是对$x(qt_n)$的近似,即设$t_{m_n-1}\leqslant qt_n<t_{m_n}$, 有
    \begin{equation*}
        \overline x_n:=\frac{t_{m_n}-qt_n}{t_{m_n}-t_{m_n-1}}x_{m_n-1}+\frac{qt_n-t_{m_n-1}}{t_{m_n}-t_{m_n-1}}x_{m_n}.
    \end{equation*}
\end{frame}
\begin{frame}{数值解的长时间有界性}
    \begin{theorem}\label{dissipativity}
        如果存在常数$a>0,a_u>a_v>0$, 使得对任何$t\geqslant 0$和$u,v\in \realset^d$成立
        \begin{equation*}
            \left<u,f(t,u,v)\right>\leqslant a-a_u \|u\|^2+a_v \|v\|^2,
        \end{equation*}
        那么
        \begin{equation*}\label{耗散性结果}
            \|x_n\|\leqslant \max\left(\left(a_u-a_v\right)^{-1/2}a^{1/2},\|x_0\|\right).
        \end{equation*}
    \end{theorem}
\end{frame}
\begin{frame}{证明概要}
    在L1数值格式两边与$x_n$作内积并结合Cauchy-Schwarz不等式得到
    \begin{equation}\label{耗散性-内积放缩}
        \begin{aligned}
            \|x_n\|^2&\leqslant \left(t_n-t_{n-1}\right)^\alpha \Bigl( \Gamma(2-\alpha) \left(a-a_u \|x_n\|^2 + a_v \|\overline x_n\|^2\right)
            \\ &\relphantom{\leqslant} + \sum_{k=0}^{n-1} \left(a_{n,k+1}-a_{n,k}\right) \frac{\|x_k\|^2+\|x_n\|^2}{2}\Bigr).
        \end{aligned}
    \end{equation}
    注意到$\sum_{k=0}^{n-1} \left(a_{n,k+1}-a_{n,k}\right)=a_{n,n}=\left(t_n-t_{n-1}\right)^{-\alpha}$, 有
    \begin{equation}\label{耗散性-求和}
        \left(t_n-t_{n-1}\right)^\alpha \sum_{k=0}^{n-1} \left(a_{n,k+1}-a_{n,k}\right) \frac{\|x_k\|^2+\|x_n\|^2}{2}\leqslant\frac{1}{2}\Bigl(\max_{0\leqslant k<n} \|x_k\|^2 + \|x_n\|^2\Bigr).
    \end{equation}
\end{frame}
\begin{frame}{证明概要}
    将\cref{耗散性-求和}代入\cref{耗散性-内积放缩}得到
    \begin{align*}
        \|x_n\|^2 &\leqslant a_n \left(a-a_u \|x_n\|^2 + a_v \|\overline x_n\|^2\right) + \max_{0\leqslant k<n} \|x_k\|^2
        \\ &\leqslant a_n \Bigl(a-a_u \|x_n\|^2 + a_v \max_{0\leqslant k\leqslant n} \|x_k\|^2\Bigr) + \max_{0\leqslant k<n} \|x_k\|^2,
    \end{align*}
    其中$a_n:=2\left(t_n-t_{n-1}\right)^\alpha \Gamma(2-\alpha)$. 现在易证
    \begin{equation*}
        \|x_n\|^2\leqslant \max\bigl((a_u-a_v)^{-1}a,\max_{0\leqslant k<n}\|x_k\|^2\bigr),\,n\in\positiveinteger,
    \end{equation*}
    最后归纳即得结论。
\end{frame}

\begin{frame}{数值解的稳定性}
    把\mainEquation 的初值条件改为$x(0)=y_0$, 用同样的数值算法(包括步长)产生数值解$\left(y_n\right)_{n=0}^\infty$和对“延迟”的近似$\left(\overline y_n\right)_{n=0}^\infty$, 并记$e_n:=y_n-x_n, \overline e_n:=\overline y_n-\overline x_n$.
    \begin{theorem}\label{stability}
        如果存在常数$b_u>b_v>0$, 使得对任何$t\geqslant 0$和$u,v,x,y\in \realset^d$成立
        \begin{equation*}
            \left\{
                \begin{aligned}
                    &\left<f(t,u,v)-f(t,x,v),u-x\right>\leqslant -b_u \|u-x\|^2,
                    \\ &\|f(t,u,v)-f(t,u,y)\|\leqslant b_v \|v-y\|,
                \end{aligned}
            \right.
        \end{equation*}
        那么
        \begin{equation*}
            \|e_n\|\leqslant \|e_0\|.
        \end{equation*}
    \end{theorem}
\end{frame}
\begin{frame}{证明概要}
    写出$\left(e_n\right)_{n=0}^\infty$满足的等式为
    \begin{equation*}
        e_n=\left(t_n-t_{n-1}\right)^\alpha\Bigl(\sum_{k=0}^{n-1}\left(a_{n,k+1}-a_{n,k}\right)e_k+\Gamma(2-\alpha)\left(f\left(t_n,y_n,\overline y_n\right)-f\left(t_n,x_n,\overline x_n\right)\right)\Bigr),
    \end{equation*}
    两边同$e_n$取内积并放缩得到
    \begin{equation*}
        \|e_n\|^2\leqslant \max_{0\leqslant k<n} \|e_k\|^2+a_n\Bigl(-b_u \|e_n\|^2+b_v \max_{0\leqslant k\leqslant n} \|e_k\|^2\Bigr).
    \end{equation*}
    其中$a_n:=2\left(t_n-t_{n-1}\right)^\alpha\Gamma(2-\alpha)$. 由此容易得到
    \begin{equation*}
        \|e_n\|\leqslant\max_{0\leqslant k< n} \|e_k\|,\,n\in\positiveinteger.
    \end{equation*}
    最后归纳即得结论。
\end{frame}

\section{后期拟完成的研究工作及进度安排}
\begin{frame}{后期拟完成的研究工作及进度安排}
    \begin{itemize}\setlength{\itemsep}{1.5em}
        \item[$\bullet$] 2024年7月,整理现有工作,并考虑向带有Caputo导数的弱奇性方程推广。
        \item[$\bullet$] 2024年8-11月,涉猎关于分数阶数值方法收敛性的文章,并考虑\mainEquation 的L1格式的收敛性。
        \item[$\bullet$] 2024年12月-2025年2月,考虑更多类型的方程和数值算法。
        \item[$\bullet$] 2025年3月-2025年5月,撰写毕业论文,准备毕业答辩。
    \end{itemize}
\end{frame}

\section{存在的困难与问题}
\begin{frame}{存在的困难与问题}
    分数阶导数的记忆性导致数值方法的收敛性难以分析。
    \\~\\
    这里数值方法收敛指的是在时间步长趋于0时,数值解和弱解的差距也会趋于0.
\end{frame}

\section*{结束致谢}

{
    \setbeamertemplate{background canvas}{\includegraphics[page=5, height=\paperheight]{template.pdf}}
    \begin{frame} \end{frame}
}

\end{document}